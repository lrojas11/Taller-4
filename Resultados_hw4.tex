\documentclass{article}
\usepackage[utf8]{inputenc}

\title{Taller 4 - Métodos computacionales}
\author{Laura Rojas Vergara}
\date{November 2018}

\usepackage{natbib}
\usepackage{graphicx}

\begin{document}

\maketitle

\section{ODE}

En este ejercicio se resolvió la ecuación de un proyectil para el caso en que hay fricción proporcional a $v^2$:

\begin{equation}
   \frac{d^2\vec{x}(t)}{dt^2} = -\vec{g}-c \frac{|\vec{v}(t)|^2}{m} \frac{\vec{v}(t)}{|\vec{v}(t)|}
\end{equation}

Donde la aceleración gravitacional g = 10 $m/s^2$, el coeficiente de fricción c = 0.2 y la masa del proyectil m = 0.2 kg. 

\subsection{Ecuación Movimiento Proyectil}

En la primera parte se tomo como condición inicial $\vec{x}(t = 0) = (0,0)$ y una velocidad de 300 m/s, con un ángulo de $45^\circ$ respecto a la horizontal. 



De esta gráfica \ref{proyectil} podemos observar el comportamiento del proyectil para los distintos ángulos. Vale la pena resaltar que el ángulo con mayor alcance horizontal es el de $20^\circ$ , mientras que para el ángulo de $45^\circ$, es aproximadamente la mitad. Adicionalmente, se plotearon todos los ángulos en una misma imagen para un mejor análisis y comparación de los resultados.

\begin{figure}[h!]
\centering
\includegraphics[scale=0.8]{Dinamica_proyectil}
\caption{Gráfica para la trayectoria del proyectil a diferentes ángulos}
\label{proyectil}
\end{figure}



%\bibliographystyle{plain}
%\bibliography{references}
\end{document}
 
